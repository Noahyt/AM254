\documentclass{article}

\usepackage{ amsmath, amssymb}
\usepackage{esvect}
\usepackage{bm}

% Tables.
\usepackage{multirow}
\usepackage{booktabs}

% Figures.
\usepackage{graphicx}
\graphicspath{{figures/}}
\usepackage{caption}
\usepackage{subcaption}

% Citation/Bibliograph.
\usepackage{natbib} % Include `\cite` command.

% Symbols
\newcommand*\Laplace{\mathop{}\!\mathbin\bigtriangleup}
\newcommand*\DAlambert{\mathop{}\!\mathbin\Box}


\title{AM254 Final Paper}
\author{Noah Toyonaga}
\date{12/15/22}

\begin{document}
\maketitle


\section{Introduction}
Deriving the normal modes of a chain of masses connected by harmonic springs is one of the 
classic problems encountered in an introductory physics curriculum.
This exercise demonstrates the elegance of 
vector/matrix notations for coupled differential equations 
and simultanously motivates the connection between discrete and continuum systems. 
This project proceeds from this classic (and classical) system
and examines what happens when we augment or replace the nearest neighbor interactions with random couplings. 
The experiments which we will explore in this work are chosen to explore two regimes of geometric disorder. 
First, in which disorder is local, but strong (compared to the mean field/nearest neighbor model), and second in which disorder is long range but perturbative. 

\begin{enumerate}
	\item \textbf{Intermediate range}: We replace nearest neighbor interactions with random strength couplings to
		neighbors within a given bandwidth. 
		This corresponds to a graph laplacian with the approximate structure of band random matrices. 
	\item \textbf{Long range}: We assume and underlying nearest neighbor interaction but perturb the system with arbitrarily long range interactions.
		In particular, we compare the behaviour of the system in the presence of long range coupling to the behavior in the finite bandwidth regime. 
\end{enumerate}

We note that the above experiments do not represent an exhaustive typology of sources of disorder in a 1D lattice model. 
Freeman Dysons's first engagement with random matrix theory, for example,
is motivated by his investigation of the spectra 
of 1D lattices in which either the stiffness of the springs or mass of nodes are varied (\cite{Dyson1953-oa,Forrester2021-xr}). 

\subsection{The ordered chain and its spectrum.}

To eluciadate the random connection models, we first review the spring-lattice model and provide a construction 
for the stiffness matrix K.
Consider the 1D chain shown in fig. \ref{fig:1D_chain} which consists of $N$ masses (labeled $m_i$) and connecting springs (where a spring between $m_i$ and $m_j$ has a stiffness $k_{i,j}$).
The force experienced by the mass $m_i$ in the deformed configuration is given by the sum of forces due to adjacent springs as given by eq \ref{eq:general_lattice_force}.

\begin{equation}\label{eq:general_lattice_force}
\begin{split}
	f_i = k_{i,i+1}&\left[ \left(x_{i+1} + u_{i+1} \right)- \left(x_{i} + u_{i} \right)  - l_i\right] \\
		 &+ k_{i-1,i}\left[ \left(x_{i-1} + u_{i-1} \right)- \left(x_{i} + u_{i} \right)  - l_{i-1}\right]
\end{split}
\end{equation}


If we assume that the system is initally in an unstressed state 
(i.e. for every $i$, $x_{i+1} - x_{i} = l_i$) then \ref{eq:general_lattice_force} simplifies to:

\begin{equation}\label{eq:lattice_force}
	f_i = k_{i,i+1}\left[u_{i+1} - u_i\right] + k_{i-1, i}\left[u_{i-1} - u_{i}\right]
\end{equation}

We finally note that the $N$ system of equations corresponding to all the masses in the system can be concisely expressed by a linear equation \ref{eq:lattice_matrix}.
Here, the $i$-th element of $\bm{F}$ and $\bm{u}$ are the force on and displacement of the mass $m_i$ respectively.
The stiffness matrix $\bm{K}_0$ is given explicitly by \ref{eq:k_lattice}.

\begin{equation}\label{eq:lattice_matrix}
	\bm{F} = \bm{K}_0\bm{u} 
\end{equation}

\begin{equation}\label{eq:k_lattice}
		\bm{K}_0 = 
\begin{pmatrix}
	-(k_{0,1} + k_{1,2}) & k_{1,2} & 0 & &\cdots & 0 \\
	k_{1,2} & -(k_{1,2} + k_{2,3}) & k_{2,3} & 0 & \cdots & 0 \\
	\vdots & &\ddots& \ddots  & \ddots & 0\\
	0& & \cdots && -(k_{N-2, N-1} + k_{N-1, N}) & k_{N-1, N}\\
	0 & & \cdots & & k_{N-1, N}& -(k_{N-1, N} + k_{N, N+1})
\end{pmatrix} 
\end{equation}

Anticipating the addition of other non-adjacent bonds, we define a more general stiffness matrix $\bm{K}$ in \ref{eq:K_general}. 
We note that $\bm{K}$ is a generalized graph laplacian of the system weighted by the relevant spring stiffnesses (\cite{Chung1997-dc}).

\begin{equation}\label{eq:K_general}
	K_{i,j}= \begin{cases}
		-1 * \sum k_{i,j} & if i=j \\
		k_{i,j} & if i
	\end{cases}
\end{equation}

The spectrum and eigenmodes of $\bm{K}_0$ are illustrated in \ref{fig:lattice_spectrum}.

\begin{figure}
\begin{center}
	\includegraphics[width=.5\textwidth]{Figures/}
\end{center}
\caption{A visual representation of the 1D chain with coordinates. 
	A. shows the (unstressed) equilibrium configuration while B. illustrates the deformed system.
The masses $m_i$ are indexed by $i \in (1, N)$, with corresponding displacements are $u_i$.}
\label{fig:1D_chain}
\end{figure}


\begin{figure}
\begin{center}
\includegraphics[width=\textwidth]{Figures/}
\end{center}
\caption{A. The spectrum of the 1D chain with stiffness matrix $\bm{K}_0$. B. 
The first and last two eigenmodes of $\bm{K}_0$. These correspond to the highest and lowest energy (frequency) excitations of the system.
C. The amplitude of the complete set of eigenmodes of the system, ordered from low to high frequency. 
Each row represents a single eigenmode.
We have chosen to plot the absolute amplitude, rather than the signed aplitude in order to highlight the degree of (de)localization. 
Indeed, in the $\bm{K}_0$ system there are no localized modes in the sense defined in \ref{sec:localization}.}
\label{fig:lattice_spectrum}
\end{figure}

\section{Localization}
\label{sec:localization}

To motivate a study of eigenmode localization, we first consider the case of \textit{Anderson localization}, 
which correspoinds to the confinement of waves in the presence of a disordered background potential (\cite{Anderson1958-dt}).
This phenomena is of interest, because it suggests that evergy can be localized in a system without confinement.\footnote{
\cite{Filoche2012-av} give an elegant proof for the localization of eigenmodes as a conequence of a set of effective contraints defined in the \textit{interior} of the domain generated by the field $V$.}
The eigenproblem for this model is given by eq \ref{eq:anderson} where the potential $V(x)$ is a random distribution over space. 

\begin{equation}\label{eq:anderson}
	\left( -\Laplace + V\right)u = \lambda u 
\end{equation}

In our experiment, we construct $V$ by partitioning the domain into (n=20) bins and setting $V$ within each bin to a random value uniformly distribitued between (0, ). 
We also normalize the matrix $\bm{K}_0$ by a factor $\frac{l}{N^2}$ so that it can be interpreted as the discrete spatial laplacian $\Laplace$ for a domain of size $l$. 
(In this and the rest of the experiments in this work we assume $l=1$.) 
A representative experiment is shown in \ref{fig:anderson_eigenvalues} were we can clearly see the spatial localization of eigenmodes.

We now introduce two metrics for the degree of localization. 
The first is inspired by (\cite{Casati1990-ma}), and corresponds to the cross entropy of the eigenmode compared to a fully delocalized mode.

The second metric is a purely heuristic computational tool, that gives the fraction 

\begin{figure}
\begin{center}
\includegraphics[width=\textwidth]{Figures/}
\end{center}
\caption{
	A. The spectrum of $\bm{K}_0 + V$ (V is defined in the text). 
	B. 
}
\label{fig:anderson_eigenvalues}
\end{figure}




\section{localization}%
\label{sec:localization}

We define a localization parameter, following Molinari.

\begin{equation}
	\beta \equiv \exp{\left( \langle H \rangle - H_{GOE} \right)}
\end{equation}
\begin{equation}
	H \left( u \right)\equiv \sum u_i ^2 \log{\left( u_i^2 \right)}
\end{equation}

\section{sparse}%
\label{sec:sparse}

\begin{equation}
	\boldsymbol{A} = \theta \nabla + X
\end{equation}

$X$ represents a contribution from random bonds. 

It is symmetric, gaussian random entries, and has 0 on the diagonal.

\bibliography{references}
\bibliographystyle{plainnat} % Needed for \cite.


\end{document}
