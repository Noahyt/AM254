\documentclass[11pt,letter]{memoir}
%%%%%% INSTRUCTIONS %%%%%%
% For scribers, please save a copy of your drafts locally!
% (Writing in this project is dangerous -- imagine someone pressing ctrl+z while you're working!)

% Upload your tex files separately, and add one line in this main doc:
%   \include{YOUR_FILE_NAME}   (no ".tex" needed)
%%%%%%

% Common commands and definitions
%% reduce spacing in para environment
\makeatletter
\renewcommand\paragraph{\@startsection{paragraph}{4}{\z@}%
                                    {1ex \@plus1ex \@minus.2ex}%
                                    {-1em}%
                                    {\normalfont\normalsize\bfseries}}
\makeatother
%%%


\usepackage{amsmath}
\usepackage{amssymb}
\usepackage{latexsym}
\usepackage{verbatim}
%\usepackage{subfigure}
\usepackage[final]{graphicx}
\usepackage{psfrag}
\usepackage{mathrsfs}  
\usepackage{amsthm}
%\usepackage{xcolor}
\usepackage{fancybox}
\usepackage{enumitem}
%\usepackage{mathabx}
\usepackage[export]{adjustbox}
\usepackage{caption,subcaption,multicol}



\newtheorem{theorem}{Theorem}
\newtheorem{proposition}{Proposition}
\newtheorem{lemma}{Lemma}
\newtheorem{corollary}{Corollary}
\newtheorem{property}{Property}
\newtheorem{conjecture}{Conjecture}
\newtheorem{condition}{Condition}
\newtheorem{procedure}{Procedure}
%\newtheorem{assumption}{Assumption}

\theoremstyle{definition}
\newtheorem{example}{Example}
\newtheorem{definition}{Definition}
\newtheorem{assumption}{Assumption}
\newtheorem{sassumption}{Simplifying Assumption}

\theoremstyle{remark}
\newtheorem{remark}{Remark}
\newtheorem{fact}{Fact}



\newenvironment{densitemize}%    Densely printed itemized list,
{\begin{list}               %    with flush left bullets.
    {$\bullet$ \hfill}{
        \setlength{\leftmargin}{\parindent}
        \setlength{\parsep}{0.04\baselineskip}
        \setlength{\itemsep}{0.5\parsep}
        \setlength{\labelwidth}{\leftmargin}
        \setlength{\labelsep}{0em}}
    }
{\end{list}}

\providecommand{\eref}[1]{\eqref{eq:#1}}  % call \eqref from amstex
\providecommand{\cref}[1]{Chapter~\ref{chap:#1}}
\providecommand{\sref}[1]{Section~\ref{sec:#1}}
\providecommand{\appref}[1]{Appendix~\ref{appendix:#1}}
\providecommand{\fref}[1]{Figure~\ref{fig:#1}}
\providecommand{\tref}[1]{Table~\ref{tab:#1}}
\providecommand{\thref}[1]{Theorem~\ref{thm:#1}}
\providecommand{\defref}[1]{Definition~\ref{def:#1}}
\providecommand{\lemref}[1]{Lemma~\ref{lem:#1}}
\providecommand{\remref}[1]{Remark~\ref{rem:#1}}
\providecommand{\assumpref}[1]{Assumption~\ref{assump:#1}}
\providecommand{\sassumpref}[1]{Simplifying Assumption~\ref{sassump:#1}}
\providecommand{\factref}[1]{Fact~\ref{fact:#1}}
\providecommand{\propref}[1]{Proposition~\ref{prop:#1}}
\providecommand{\corref}[1]{Corollary~\ref{cor:#1}}
\providecommand{\exref}[1]{Example~\ref{ex:#1}}

\providecommand{\R}{\ensuremath{\mathbb{R}}}
\providecommand{\C}{\ensuremath{\mathbb{C}}}
\providecommand{\N}{\ensuremath{\mathbb{N}}}
\providecommand{\Z}{\ensuremath{\mathbb{Z}}}
\providecommand{\W}{\ensuremath{\mathbb{N}_0}}


\providecommand{\bfP}{\mathbf{P}}
\providecommand{\abs}[1]{\lvert#1\rvert}
\providecommand{\norm}[1]{\lVert#1\rVert}
\providecommand{\inprod}[1]{\langle#1\rangle}
\providecommand{\set}[1]{\left\{#1\right\}}
\providecommand{\seq}[1]{\left<#1\right>}
\providecommand{\bydef}{\overset{\text{def}}{=}}
\renewcommand{\dim}{N}
\newcommand{\ssize}{M}
\newcommand{\reg}{\rho}
\newcommand{\uclass}[1]{\mathscr{U}(#1)}
\newcommand{\auxmat}{\mat{A}}
\newcommand{\auxvec}{{a}}
\newcommand{\auxdim}{b}
\newcommand{\order}{k}
\newcommand{\opt}{\mathrm{RLS}}
\providecommand{\sigpsi}{\sigma_{\psi}^2}
\providecommand{\siglam}{\sigma_{\lambda}^2}

\renewcommand{\vec}[1]{\ensuremath{\boldsymbol{#1}}}
\providecommand{\mat}[1]{\ensuremath{\boldsymbol{#1}}}

% Some calligraphic letters
\providecommand{\calR}{\mathcal{R}}
\providecommand{\calS}{\mathcal{S}}
\providecommand{\calP}{\mathcal{P}}
\providecommand{\calH}{\mathcal{H}}
\providecommand{\calM}{\mathcal{M}}
\providecommand{\calU}{\mathcal{U}}
\providecommand{\calV}{\mathcal{V}}
\providecommand{\calF}{\mathcal{F}}
\providecommand{\calO}{\mathcal{O}}

% Common matrices and vectors
\providecommand{\mA}{\mat{A}} \providecommand{\mB}{\mat{B}}
\providecommand{\mC}{\mat{C}} 
\providecommand{\mD}{\mat{D}}
\providecommand{\mF}{\mat{F}}
\providecommand{\mH}{\mat{H}}
\providecommand{\mI}{\mat{I}} \providecommand{\mJ}{\mat{J}} 
\providecommand{\mK}{\mat{K}} \providecommand{\mL}{\mat{L}} 
\providecommand{\mM}{\mat{M}} \providecommand{\mP}{\mat{P}} 
\providecommand{\mQ}{\mat{Q}} \providecommand{\mR}{\mat{R}}
\providecommand{\mS}{\mat{S}} \providecommand{\mU}{\mat{U}} 
\providecommand{\mV}{\mat{V}}
\providecommand{\mW}{\mat{W}}
\providecommand{\mT}{\mat{T}}
\providecommand{\mZ}{\mat{Z}}
\providecommand{\mSigma}{\mat{\Sigma}}
\providecommand{\mGm}{\mat{\Gamma}} \providecommand{\mG}{\mat{G}}
\providecommand{\mPsi}{\mat{\Psi}}
\providecommand{\mPsi}{\mat{\Psi}}
\providecommand{\mX}{\mat{X}}
\providecommand{\mE}{\mat{E}}
\providecommand{\mXi}{\mat{\Xi}}
\providecommand{\linmap}{\mathscr{M}}
\providecommand{\mO}{\mat{O}}



\providecommand{\va}{\vec{a}} \providecommand{\vb}{\vec{b}}
\providecommand{\vc}{\vec{c}} \providecommand{\ve}{\vec{e}}
\providecommand{\vg}{\vec{g}}
\providecommand{\vh}{\vec{h}} \providecommand{\vk}{\vec{k}}
\providecommand{\vm}{\vec{m}} \providecommand{\vn}{\vec{n}} 
\providecommand{\vl}{\vec{l}} \providecommand{\vp}{\vec{p}}
\providecommand{\vq}{\vec{q}} 
\providecommand{\vr}{\vec{r}}
\providecommand{\vs}{\vec{s}}
\providecommand{\vt}{\vec{t}}
\providecommand{\vu}{\vec{u}} \providecommand{\vw}{\vec{w}}
\providecommand{\vx}{\vec{x}} \providecommand{\vy}{\vec{y}}
\providecommand{\vz}{\vec{z}} \providecommand{\vi}{\vec{i}}
\providecommand{\vj}{\vec{j}} \providecommand{\vzero}{\vec{0}}
\providecommand{\vgm}{\vec{\gamma}} 
\providecommand{\vxi}{\vec{\xi}}
\providecommand{\vv}{\vec{v}}
\providecommand{\vomg}{\vec{\omega}}
\providecommand{\vell}{\vec{\ell}}
\providecommand{\vDelta}{\vec{\Delta}}
\providecommand{\valpha}{\vec{\alpha}}
\providecommand{\vbeta}{\vec{\beta}}
\providecommand{\vepsilon}{\vec{\epsilon}}
\providecommand{\vsigma}{\vec{\sigma}}
\providecommand{\fieldvec}{\vec{1}}
\providecommand{\fieldscal}{{s}}
\providecommand{\vone}{\vec{1}}
\providecommand{\vgamma}{\vec{\gamma}}
\providecommand{\vdelta}{\vec{\delta}}
\newcommand{\auxC}{L}



\providecommand{\w}{\omega}

\providecommand{\conv}{\ast}
\providecommand{\circonv}[1]{\circledast_{#1}}

%%% additional commands
\newcommand{\unif}[1]{\mathsf{Unif}(#1)} %% use for Haar measure/ Uniform measure
\newcommand{\GOE}[1]{\mathsf{GOE}(#1)} 
\newcommand{\ortho}{\mathbb{O}} %% use for orthogonal group
\newcommand{\explain}[2]{\overset{\text{\tiny{#1}}}{#2}}
\newcommand{\iter}[2]{{#1}^{(#2)}} %% use it to define iterates
\newcommand{\E}{\mathbb{E}} %% for expectations
\renewcommand{\P}{\mathbb{P}} %% for probabilities
\newcommand{\Var}{\mathrm{Var}}
\usepackage[sort&compress,numbers]{natbib} % for citet, citep etc
\newcommand{\sphere}[1]{\mathcal{S}^{#1}} % use for unit sphere
\newcommand{\ip}[2]{\left\langle {#1}, {#2} \right\rangle} %use for inner product
\newcommand{\Op}{\mathcal{O}_\mathsf{P}} %% for stochastic order notation
\newcommand{\gauss}[2]{\mathcal{N}\left( #1,#2 \right)} %% for Gaussian distribution
\newcommand{\nonlin}{f}

%% commands related to combinatorics
\newcommand{\degree}{D}
\newcommand{\fourier}[3]{\hat{\nonlin}(#1,#2; #3)}
\newcommand{\pweight}[2]{p_{#2}(#1)}
\newcommand{\qweight}[2]{q_{#2}({#1})}
\newcommand{\effect}[1]{\omega_{#1}}
\newcommand{\height}[2][]{h_{#1}{(#2)}}
\newcommand{\valid}{\mathtt{VALID}}
\newcommand{\relev}{\mathtt{RELEV}}
\newcommand{\hermite}[1]{H_{#1}}
\newcommand{\trees}[2]{\mathscr{T}_{#1}(#2)}
\newcommand{\colorings}[2]{\mathscr{C}_{#1}(#2)}
\newcommand{\leaves}[1]{\mathscr{L}(#1)}
\newcommand{\nleaves}{\mathscr{L}_0}
\renewcommand{\part}[1]{\mathscr{P}(#1)}
\newcommand{\chrn}[1]{c_{#1}}
\newcommand{\pw}{$\mathrm{PW}_2$}
\newcommand{\serv}[1]{\mathsf{#1}}
\newcommand{\nullleaves}[1]{\mathscr{L}_0(#1)}
\newcommand{\exponent}{\eta}
\newcommand{\noisestd}{\sigma}

\DeclareMathOperator*{\argmin}{\arg\min}

\makeatletter
\newcommand{\myitem}[1]{%
\item{#1} \protected@edef\@currentlabel{#1}%
}
\makeatother

\makeatletter
\newenvironment{tagsubequations}[1]
 {%
  \addtocounter{equation}{-1}%
  \begin{subequations}
  \renewcommand{\theparentequation}{#1}%
  \def\@currentlabel{#1}%
 }
 {%
  \end{subequations}\ignorespacesafterend
 }
\makeatother


%%%% for algorithms
%%% for algorithms

\newenvironment{fminipage}%
  {\begin{Sbox}\begin{minipage}}%
  {\end{minipage}\end{Sbox}\fbox{\TheSbox}}

\newenvironment{algbox}[0]{\vskip 0.2in
\noindent 
\begin{fminipage}{6.3in}
}{
\end{fminipage}
\vskip 0.2in
}

%% code for widecheck
\DeclareFontFamily{U}{mathx}{\hyphenchar\font45}
\DeclareFontShape{U}{mathx}{m}{n}{
      <5> <6> <7> <8> <9> <10>
      <10.95> <12> <14.4> <17.28> <20.74> <24.88>
      mathx10
      }{}
\DeclareSymbolFont{mathx}{U}{mathx}{m}{n}
\DeclareFontSubstitution{U}{mathx}{m}{n}
\DeclareMathAccent{\widecheck}{0}{mathx}{"71}
\DeclareMathAccent{\wideparen}{0}{mathx}{"75}

\def\cs#1{\texttt{\char`\\#1}}
%%%%%%%
\providecommand{\ctr}[1]{\widecheck{#1}}
\renewcommand{\tilde}{\widetilde}
\renewcommand{\hat}{\widehat}
\providecommand{\asymeq}{\explain{\pw}{\simeq}}


\usepackage[utf8]{inputenc}
\usepackage[T1]{fontenc}
\usepackage{microtype}
\usepackage{xcolor}
\usepackage{times}

\usepackage{color,calc,graphicx,soul,fourier}
\definecolor{nicered}{rgb}{.647,.129,.149}
\makeatletter
\newlength\dlf@normtxtw
\setlength\dlf@normtxtw{\textwidth}
\def\myhelvetfont{\def\sfdefault{mdput}}
\newsavebox{\feline@chapter}
\newcommand\feline@chapter@marker[1][4cm]{%
  \sbox\feline@chapter{%
    \resizebox{!}{#1}{\fboxsep=1pt%
      \colorbox{nicered}{\color{white}\bfseries\sffamily\thechapter}%
    }}%
  \rotatebox{90}{%
    \resizebox{%
      \heightof{\usebox{\feline@chapter}}+\depthof{\usebox{\feline@chapter}}}%
    {!}{\scshape\so\@chapapp}}\quad%
  \raisebox{\depthof{\usebox{\feline@chapter}}}{\usebox{\feline@chapter}}%
}
\newcommand\feline@chm[1][4cm]{%
  \sbox\feline@chapter{\feline@chapter@marker[#1]}%
  \makebox[0pt][l]{% aka \rlap
    \makebox[1cm][r]{\usebox\feline@chapter}%
  }}
\makechapterstyle{daleif1}{
  \renewcommand\chapnamefont{\normalfont\Large\scshape\raggedleft\so}
  \renewcommand\chaptitlefont{\normalfont\huge\bfseries\scshape\color{nicered}}
  \renewcommand\chapternamenum{}
  \renewcommand\printchaptername{}
  \renewcommand\printchapternum{\null\hfill\feline@chm[2.5cm]\par}
  \renewcommand\afterchapternum{\par\vskip\midchapskip}
  \renewcommand\printchaptertitle[1]{\chaptitlefont\raggedleft ##1\par}
}
\makeatother
\chapterstyle{daleif1}

\begin{document}

%%%---%%%---%%%---%%%---%%%---%%%---%%%---%%%---%%%---%%%---%%%---%%%---%%%
%   TITLEPAGE
%
%   due to variety of titlepage schemes it is probably better to make titlepage manually
%
%%%---%%%---%%%---%%%---%%%---%%%---%%%---%%%---%%%---%%%---%%%---%%%---%%%
\thispagestyle{empty}

{%%%
\sffamily
\centering
\Large

~\vspace{\fill}

{\huge \color{nicered} \bfseries
Mathematics of High-Dimensional Information Processing and Learning
}

\vspace{2.5cm}

{\LARGE \bfseries
Lecture Notes
}

\vspace{2ex}
{(This version: \today)}

\vspace{3.5cm}
AM/ES 254\\
Harvard University\\
Fall 2022

\vspace{3.5cm}

Instructor: Yue M. Lu\\


%%%
}%%%

\clearpage
%%%---%%%---%%%---%%%---%%%---%%%---%%%---%%%---%%%---%%%---%%%---%%%---%%%
%%%---%%%---%%%---%%%---%%%---%%%---%%%---%%%---%%%---%%%---%%%---%%%---%%%

\tableofcontents*

\clearpage

Disclaimer: These scribe notes have not been subjected to the usual scrutiny reserved for formal publications.

\include{probability_review}

\include{random_matrices}
\chapter{Gaussian Comparisons}


\section{CGMT}

Setup (Review from last time)

Suppose we have $\{X_i\}$ generated from the following process, where $y_i \in \{-1, 1\}$, $|\xi| = 1$, $\sigma$ known.

$$X_i = y_i \xi + \sigma Z_i$$

It is helpful to also define a few additional constants. First, if we have $n$ samples and dimension $d$, let $\alpha = \frac{n}{d}$ and define $P(Y_i = 1) = r$. In this section, we are using a linear function as a separator of the form 


$$\text{sign}(\frac{X_i^Tw}{\sqrt{d}} + b)$$

The goal is to learn $w, b$ so that we can minimize the error of new samples generated from the same distribution. In other words, if $X_{new}, y_{new}$ are from the same process, the goal is to learn $w^*, B^*$ to minimize

\begin{align*}
    \xi_{gen} &= \P(y_{new} \ne \text{sign}\big(\frac{X_{new}^Tw^*}{\sqrt{d}} + b^*\big) \\
    &= rQ\frac{b+1}{\tau} + (1-r)Q\frac{q-b}{\sigma \tau}
\end{align*}

Note that we defined two more (very) useful constants above, 

$$\tau := \frac{||w^*||}{\sqrt{d}}$$

$$q := \frac{\xi^Tw^*}{\sqrt{d}}$$

and the function $Q$ is defined as 

$$Q(x) := \frac{1}{\sqrt{2\pi}}\int_{x}^\infty e^{-t^2/2}dt$$

To summarize from last class, for a given $\lambda, \ell()$, we want to find $w^*, b^*$ such that


$$w^*, b^* = argmin_{w,b} \frac{1}{d}\sum_{i=1}^n \ell(\frac{x_i^Tw}{\sqrt{d}} + b; y_i) + \frac{\lambda ||w||^2}{2d}$$

The trick observed previously is that we can turn this into the following "minimax" problem.


\begin{align*}
    &\min_{w,b} \max_{u \in \mathbb{R}^n} \frac{1}{d}\sum_{i=1}^n u_i\Big(\frac{\xi^Tw}{\sqrt{d}} + \frac{\sigma Z_i^T w}{\sqrt{d}} + by_i\Big) - \frac{1}{d}\sum_{i=1}^n \ell^*(u_i) + \frac{\lambda||w||^2}{2d} \\
    = &\min_{w,b} \max_{u \in \mathbb{R}^n} \frac{\sigma}{d\sqrt{d}} u^TZw + \frac{1}{d}\sum_{i=1}^n u_i\Big(\frac{\xi^Tw}{\sqrt{d}} + by_i\Big)- \frac{1}{d}\sum_{i=1}^n \ell^*(u_i) + \frac{\lambda||w||^2}{2d}
\end{align*}

Note that the last three terms are a function $f(w,u)$ that is convex with respect to $w$ while concave with respect to $u$. Applying the CGMT, we can rewrite this in the form 

\begin{align*}
    \min_{w,b} \max_{u \in \mathbb{R}^n} \frac{\sigma}{d\sqrt{d}} ||u||s^Tw + \frac{\sigma}{d\sqrt{d}} ||w||u^Tg + \frac{1}{d}\sum_{i=1}^n u_i\Big(\frac{\xi^Tw}{\sqrt{d}} + by_i\Big)- \frac{1}{d}\sum_{i=1}^n \ell^*(u_i) + \frac{\lambda||w||^2}{2d}
\end{align*}

A useful "trick" is to add an additional minimization to the above problem, where we use $\tau = \frac{||w||}{\sqrt{d}}$ and $q = \frac{\xi^Tw}{\sqrt{d}}$. Adding this additional minimization simplifies the functional form significantly, to give

\begin{align*}
    \min_{\tau, q} \min_{\frac{||w||}{\sqrt{d}} = \tau, \frac{\xi^Tw}{\sqrt{d}} = q} \max_{u \in \mathbb{R}^n} \frac{\sigma||u||}{d\sqrt{d}} s^Tw + \frac{\sigma\tau}{d} u^Tg + \frac{1}{d}U^T(q + by) - \frac{1}{d}\sum_{i=1}^n \ell^*(u_i) + \frac{\lambda\tau^2}{2}
\end{align*}

This was useful because now fewer terms depend on $w$.

Now, we can rewrite

$$s = |s^T\xi|\xi + P_{\xi^{\perp}}(s)$$

and simplify the fact that 

\begin{align*}
    s^tw/\sqrt{d} &= \big(|s^T\xi|\xi + P_{\xi^{\perp}}(s) \big)\big(q\xi + P_{\xi^{\perp}}(w)/\sqrt{d}\big) \\
    &= qs^T\xi + P_{\xi^{\perp}}(s)^TP_{\xi^{\perp}}(w)/\sqrt{d} \\
    &= qs^T\xi - \sqrt{\tau^2 - q^2}\sqrt{(||s||^2 - (s^T\xi)^2)/d}
\end{align*}

 Now we have not yet done anything probabilistic. However, we are looking in the case of when $d$ is large (high dimensional), therefore we can ignore the term $qs^T\xi$. This is becauseq $q$ is just a scalar, $s^T\xi$ is just a standard gaussian, so with high probability is order 1. Therefore, because we have a $\frac{1}{\sqrt{d}}$ coefficient in front of this, it goes to $0$ and we can remove this term.
 
 Now, there is another term we can consider, namely 
 
 
 $$\frac{||s||^2 - (s^T\xi)^2}{d} \longrightarrow 1$$
 
 This is because by the LLN, we have that
 
 $$\frac{||s||^2}{d} = \frac{\sum_{i=1}^d s_i^2}{d } = 1$$
 
 we can rigorously do this with error term and stochastic domination, but this is essentially $1$. The term $(s^T\xi)/d \longrightarrow 0$ by the same logic as above.
 
 
 Remember that we have the special case when: 
 
 
 $$\ell(x) = (x-1)^2$$ 
 
 $$\ell^*(u) = u^2/2 + u$$
  With these simplifications, we can rewrite our entire problem as the following.

 
 \begin{align*}
     \min_{\tau, q} \max_{u} \frac{-\sigma||u||}{d}\sqrt{\tau^2 - q^2} + \frac{1}{d}U^T(\sigma \tau g + q + by - 1) - \frac{1}{d}||u||^2 + \frac{\lambda\tau^2}{2}
 \end{align*}
 
 This problem looks much more promising, but we still have to deal with the $u$ vector which is still high dimensional. How can we simplify $u$? The nice thing is that $u$ appears as an inner product and as a norm in the expression.

\appendix

\chapter{The Householder Matrix}
Lorem ipsum dolor sit amet, consectetur adipiscing elit, sed do eiusmod tempor incididunt ut labore et dolore magna aliqua. Ut enim ad minim veniam, quis nostrud exercitation ullamco laboris nisi ut aliquip ex ea commodo consequat. Duis aute irure dolor in reprehenderit in voluptate velit esse cillum dolore eu fugiat nulla pariatur. Excepteur sint occaecat cupidatat non proident, sunt in culpa qui officia deserunt mollit anim id est laborum.


\bibliographystyle{unsrt}
\bibliography{AM254}

\end{document}

