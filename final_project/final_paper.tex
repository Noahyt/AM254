\documentclass{article}

\usepackage{ amsmath, amssymb}
\usepackage{esvect}




\title{AM254 Final Paper}
\author{Noah Toyonaga}
\date{12/15/22}

\begin{document}
\maketitle


\section{Introduction}
Deriving the normal modes of a chain of masses connected by harmonic springs is one of the 
classic problems encountered in an introductory physics curriculum.
In one fell swoop this exercise both demonstrates the elegance of 
vector/matrix notations for coupled differential equations 
and simultanously motivates the connection between discrete and continuum systems. 
This project proceeds from this classic (and classical) system
and examines what happens when we augment or replace the nearest neighbor interactions with random couplings. 
In particular we will explore two regimes:

\begin{enumerate}
	\item \textbf{Intermediate range}: We replace nearest neighbor interactions with random strength couplings to
		neighbors within a given bandwidth. 
		This corresponds to the case of connectivity with the structure of band random matrices. 
	\item \textbf{Long range}: We preserve the NN interaction, but perturb the system with arbitrarily long range interactions.
		In particular, we compare the behaviour of the system in the presence of long range coupling to the behavior in the finite bandwidth regime. 
\end{enumerate}




\section{Random Spring Networks}

\begin{equation}
	\text{Harmonic Spring: } \quad F = -k u \quad \left( u\equiv x - x_0\right)
\end{equation}


\begin{equation}
	\text{Newton's Equation} \quad F = m a = m u_{tt} 
\end{equation}

\begin{equation}
	\vec{u}_{tt} = \boldsymbol{A}\vec{u}
\end{equation}

\begin{equation}
	\boldsymbol{A}_0 = 
\begin{pmatrix}
	-2 & 1 & 0 & &\cdots & 0 \\
	1 & -2 & 1 & 0 & \cdots & 0 \\
	0 & 1 & -2 & 1 & \cdots & 0  \\
	\vdots & &\ddots& \ddots  & \ddots & 0\\
	0& & \cdots && -2 & 1\\
	0 & & \cdots & &1& -2
\end{pmatrix}
\end{equation}


\section{Poisson Equation}

\begin{equation}
	\boldsymbol{A}_0 = 
\begin{pmatrix}
	-2 & 1 & 0 & &\cdots & 0 \\
	1 & -2 & 1 & 0 & \cdots & 0 \\
	0 & 1 & -2 & 1 & \cdots & 0  \\
	\vdots & &\ddots& \ddots  & \ddots & 0\\
	0& & \cdots && -2 & 1\\
	0 & & \cdots & &1& -2
\end{pmatrix} \equiv \nabla_d
\end{equation}


\begin{equation}
	\left( -\nabla + V(\vec{x})\right) \psi = E\psi
\end{equation}

\begin{equation}
	A = \underbrace{A_0}_{\nabla} + \underbrace{\sigma\xi \xi^\intercal}_{V(x)}
\end{equation}

\section{localization}%
\label{sec:localization}

We define a localization parameter, following Molinari.

\begin{equation}
	\beta \equiv \exp{\left( \langle H \rangle - H_{GOE} \right)}
\end{equation}
\begin{equation}
	H \left( u \right)\equiv \sum u_i ^2 \log{\left( u_i^2 \right)}
\end{equation}

\section{sparse}%
\label{sec:sparse}

\begin{equation}
	\boldsymbol{A} = \theta \nabla + X
\end{equation}

$X$ represents a contribution from random bonds. 

It is symmetric, gaussian random entries, and has 0 on the diagonal.

\end{document}
